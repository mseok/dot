% Scientific Report Template
% Uses scientific_report.sty for professional formatting
% Compile with: xelatex or lualatex
%
% This template demonstrates the full capabilities of the scientific_report.sty
% style package for creating professional research reports, technical reports,
% and white papers.

\documentclass[11pt,letterpaper]{report}

% Load the scientific report style package
\usepackage{scientific_report}

% Additional packages (optional, add as needed)
\usepackage{lipsum}  % For placeholder text - remove in actual documents

% ============================================================================
% DOCUMENT METADATA
% ============================================================================

\hypersetup{
    pdftitle={Research Report Title},
    pdfauthor={Author Name},
    pdfsubject={Research Subject},
    pdfkeywords={keyword1, keyword2, keyword3}
}

% ============================================================================
% DOCUMENT BEGIN
% ============================================================================

\begin{document}

% ----------------------------------------------------------------------------
% TITLE PAGE
% ----------------------------------------------------------------------------

\makereporttitle
    {Research Report Title}                    % Title
    {A Comprehensive Analysis of [Topic]}      % Subtitle
    {Author Name, PhD}                         % Author(s)
    {Research Institution / Organization}     % Institution
    {January 2025}                             % Date

% Alternative: Title page with cover image
% \makereporttitlewithimage
%     {Research Report Title}
%     {A Comprehensive Analysis of [Topic]}
%     {../figures/cover_image.png}
%     {Author Name, PhD}
%     {Research Institution / Organization}
%     {January 2025}

% ----------------------------------------------------------------------------
% FRONT MATTER
% ----------------------------------------------------------------------------

\pagenumbering{roman}

% Table of Contents
\tableofcontents
\newpage

% List of Figures
\listoffigures
\newpage

% List of Tables
\listoftables
\newpage

% ----------------------------------------------------------------------------
% EXECUTIVE SUMMARY
% ----------------------------------------------------------------------------

\chapter*{Executive Summary}
\addcontentsline{toc}{chapter}{Executive Summary}

\begin{executivesummary}[Report Overview]
This report presents a comprehensive analysis of [topic]. Our investigation reveals significant findings that have implications for [field/application]. The research was conducted using [methodology] and involved [sample/data description].
\end{executivesummary}

\subsection*{Key Findings}

\begin{keyfindings}
\begin{enumerate}
    \item \textbf{Finding 1:} Description of the first major finding with supporting data (\pvalue{0.001}, \samplesize{250}).
    \item \textbf{Finding 2:} Description of the second major finding with effect size (\effectsize{d}{0.75}).
    \item \textbf{Finding 3:} Description of the third major finding (\meansd{42.5}{8.3}).
\end{enumerate}
\end{keyfindings}

\subsection*{Recommendations}

\begin{recommendations}
\begin{enumerate}
    \item Implement [specific action] to address [issue].
    \item Allocate resources for [specific initiative].
    \item Conduct follow-up research on [topic].
\end{enumerate}
\end{recommendations}

\newpage
\pagenumbering{arabic}

% ----------------------------------------------------------------------------
% CHAPTER 1: INTRODUCTION
% ----------------------------------------------------------------------------

\chapter{Introduction}

\section{Background}

This section provides context for the research. Include relevant background information, the current state of knowledge, and the rationale for the study.

\begin{definition}[Key Term]
A \textbf{key term} is defined as [definition]. This concept is fundamental to understanding the research presented in this report.
\end{definition}

\section{Research Objectives}

The primary objectives of this research are:

\begin{enumerate}
    \item To investigate [objective 1]
    \item To analyze [objective 2]
    \item To evaluate [objective 3]
\end{enumerate}

\section{Hypotheses}

\begin{hypothesis}[Primary Hypothesis]
We hypothesize that [independent variable] will have a significant effect on [dependent variable], such that [expected direction of effect].
\end{hypothesis}

\section{Scope and Limitations}

\begin{limitations}[Study Scope]
This research focuses on [specific scope]. The following limitations should be considered when interpreting the results:
\begin{itemize}
    \item Geographic limitation: [description]
    \item Temporal limitation: [description]
    \item Sample limitation: [description]
\end{itemize}
\end{limitations}

% ----------------------------------------------------------------------------
% CHAPTER 2: METHODOLOGY
% ----------------------------------------------------------------------------

\chapter{Methodology}

\section{Study Design}

\begin{methodology}[Research Design Overview]
This study employed a [type] design to investigate [research question]. The methodology was selected based on [rationale] and follows established guidelines for [type of research].
\end{methodology}

\section{Participants}

A total of \samplesize{500} participants were recruited from [population]. Table~\ref{tab:demographics} presents the demographic characteristics of the sample.

\begin{table}[htbp]
\centering
\caption{Participant Demographics}
\label{tab:demographics}
\begin{tabular}{@{}lcc@{}}
\toprule
\tableheader{Characteristic} & \tableheader{n} & \tableheader{\%} \\
\midrule
\textbf{Gender} & & \\
\quad Male & 245 & 49.0 \\
\rowcolor{tablealt} \quad Female & 255 & 51.0 \\
\textbf{Age Group} & & \\
\quad 18--30 & 150 & 30.0 \\
\rowcolor{tablealt} \quad 31--45 & 200 & 40.0 \\
\quad 46--60 & 120 & 24.0 \\
\rowcolor{tablealt} \quad 60+ & 30 & 6.0 \\
\bottomrule
\end{tabular}
\end{table}

\section{Instruments}

\subsection{Primary Measures}

The following instruments were used to collect data:

\begin{itemize}
    \item \textbf{Instrument 1:} Description and psychometric properties
    \item \textbf{Instrument 2:} Description and psychometric properties
    \item \textbf{Instrument 3:} Description and psychometric properties
\end{itemize}

\section{Procedures}

Data collection occurred between [dates]. The procedure involved the following steps:

\begin{enumerate}
    \item Informed consent was obtained from all participants.
    \item Baseline assessments were administered.
    \item Intervention/treatment was delivered (if applicable).
    \item Follow-up assessments were conducted at [timepoints].
\end{enumerate}

\section{Statistical Analysis}

Data were analyzed using [software] version [X.X]. The following analyses were performed:

\begin{itemize}
    \item Descriptive statistics for all variables
    \item [Analysis type] to test hypothesis 1
    \item [Analysis type] to test hypothesis 2
\end{itemize}

Statistical significance was set at $\alpha = 0.05$. Effect sizes were calculated using [method] and interpreted according to [guidelines].

% ----------------------------------------------------------------------------
% CHAPTER 3: RESULTS
% ----------------------------------------------------------------------------

\chapter{Results}

\section{Descriptive Statistics}

Table~\ref{tab:descriptives} presents the descriptive statistics for the primary outcome measures.

\begin{table}[htbp]
\centering
\caption{Descriptive Statistics for Primary Outcome Measures}
\label{tab:descriptives}
\begin{tabular}{@{}lccccc@{}}
\toprule
\tableheader{Variable} & \tableheader{n} & \tableheader{Mean} & \tableheader{SD} & \tableheader{Min} & \tableheader{Max} \\
\midrule
Outcome 1 & 500 & 42.5 & 8.3 & 18 & 72 \\
\rowcolor{tablealt} Outcome 2 & 498 & 3.7 & 1.2 & 1 & 7 \\
Outcome 3 & 495 & 128.4 & 24.7 & 68 & 195 \\
\rowcolor{tablealt} Outcome 4 & 500 & 0.68 & 0.15 & 0.28 & 0.95 \\
\bottomrule
\end{tabular}
\figurenote{SD = Standard Deviation}
\end{table}

\section{Primary Analyses}

\subsection{Hypothesis 1}

\begin{resultsbox}[Primary Finding]
Analysis revealed a significant effect of [IV] on [DV], \effectsize{F(2, 497)}{12.45}, \psig{< 0.001}, $\eta^2$ = 0.048. Post-hoc comparisons indicated that [specific findings].
\end{resultsbox}

The results support our primary hypothesis. As shown in Figure~\ref{fig:main_results}, participants in the experimental condition demonstrated significantly higher scores compared to the control group.

% Placeholder for figure
\begin{figure}[htbp]
\centering
\fbox{\parbox{0.8\textwidth}{\centering\vspace{3cm}[Figure: Main Results Comparison]\\Bar chart showing group differences\vspace{3cm}}}
\caption{Comparison of Outcome Scores by Experimental Condition}
\label{fig:main_results}
\figuresource{Study data}
\end{figure}

\subsection{Hypothesis 2}

Results indicated a moderate positive correlation between [variable 1] and [variable 2], \effectsize{r}{0.45}, \psig{< 0.001}, \CI{0.38}{0.52}.

\section{Secondary Analyses}

Table~\ref{tab:regression} presents the results of the regression analysis predicting [outcome].

\begin{table}[htbp]
\centering
\caption{Multiple Regression Analysis Predicting Outcome}
\label{tab:regression}
\begin{tabular}{@{}lcccc@{}}
\toprule
\tableheader{Predictor} & \tableheader{B} & \tableheader{SE} & \tableheader{$\beta$} & \tableheader{p} \\
\midrule
(Intercept) & 12.45 & 2.34 & --- & < .001 \\
\rowcolor{tablealt} Predictor 1 & 0.58 & 0.12 & 0.28\sigthree & < .001 \\
Predictor 2 & 0.34 & 0.09 & 0.22\sigtwo & .002 \\
\rowcolor{tablealt} Predictor 3 & 0.15 & 0.11 & 0.08\signs & .172 \\
Predictor 4 & -0.42 & 0.14 & -0.18\sigtwo & .003 \\
\midrule
\multicolumn{5}{l}{$R^2$ = 0.34, Adjusted $R^2$ = 0.33, \effectsize{F(4, 495)}{63.82}\sigthree} \\
\bottomrule
\end{tabular}

\vspace{0.5em}
{\small \siglegend}
\end{table}

% ----------------------------------------------------------------------------
% CHAPTER 4: DISCUSSION
% ----------------------------------------------------------------------------

\chapter{Discussion}

\section{Summary of Findings}

\begin{keyfindings}[Key Takeaways]
\begin{enumerate}
    \item The primary hypothesis was \highlight{supported}, with significant effects observed for [variables].
    \item Evidence quality for the main findings is \evidencestrong, based on effect sizes and replication potential.
    \item Unexpected finding: [description of any surprising results].
\end{enumerate}
\end{keyfindings}

\section{Interpretation}

The findings from this study contribute to our understanding of [topic] in several important ways. First, the significant effect of [IV] on [DV] suggests that [interpretation]. This aligns with previous research by [Author] (Year), who found similar patterns in [context].

\begin{pullquote}
``These findings represent a significant advancement in our understanding of [topic] and have direct implications for [application/practice].''
\end{pullquote}

Second, the correlation between [variables] indicates that [interpretation]. This relationship has important implications for [field/practice].

\section{Implications}

\subsection{Theoretical Implications}

The results have several theoretical implications:

\begin{itemize}
    \item Support for [theory/model]
    \item Extension of [existing framework]
    \item Challenge to [alternative perspective]
\end{itemize}

\subsection{Practical Implications}

\begin{recommendations}[Practical Applications]
Based on our findings, we recommend the following practical applications:
\begin{enumerate}
    \item \textbf{For practitioners:} [specific recommendation]
    \item \textbf{For policymakers:} [specific recommendation]
    \item \textbf{For researchers:} [specific recommendation]
\end{enumerate}
\end{recommendations}

\section{Limitations}

\begin{limitations}[Study Limitations]
Several limitations should be considered when interpreting these findings:
\begin{itemize}
    \item \textbf{Sample:} The sample was drawn from [population], which may limit generalizability to [other populations].
    \item \textbf{Design:} The [cross-sectional/correlational] design precludes causal inference.
    \item \textbf{Measurement:} Self-report measures may be subject to [bias type].
\end{itemize}
\end{limitations}

\section{Future Directions}

Future research should address the following questions:

\begin{enumerate}
    \item Does the effect generalize to [different population/context]?
    \item What mechanisms underlie the relationship between [variables]?
    \item How do [moderating factors] influence the observed effects?
\end{enumerate}

% ----------------------------------------------------------------------------
% CHAPTER 5: CONCLUSIONS
% ----------------------------------------------------------------------------

\chapter{Conclusions}

\begin{executivesummary}[Conclusion]
This research investigated [topic] using [methodology] with a sample of \samplesize{500} participants. The findings demonstrate that [main conclusion]. These results have important implications for [field/practice] and suggest that [recommendation]. Future research should continue to explore [direction].
\end{executivesummary}

\section{Key Contributions}

\begin{keyfindings}[Research Contributions]
This study makes the following key contributions:
\begin{enumerate}
    \item \textbf{Empirical contribution:} First demonstration of [finding] in [context].
    \item \textbf{Methodological contribution:} Novel application of [method] to study [phenomenon].
    \item \textbf{Practical contribution:} Evidence-based recommendations for [application].
\end{enumerate}
\end{keyfindings}

% ----------------------------------------------------------------------------
% REFERENCES
% ----------------------------------------------------------------------------

\chapter*{References}
\addcontentsline{toc}{chapter}{References}

% If using BibTeX:
% \bibliographystyle{apalike}
% \bibliography{references}

% Manual references for template demonstration:
\noindent
\hangindent=0.5in Author, A. A. (Year). Title of article. \textit{Journal Name}, \textit{Volume}(Issue), pages. https://doi.org/xxxxx

\vspace{0.5em}
\noindent
\hangindent=0.5in Author, B. B., \& Author, C. C. (Year). \textit{Title of book}. Publisher.

% ----------------------------------------------------------------------------
% APPENDICES
% ----------------------------------------------------------------------------

\appendix

\chapter{Supplementary Materials}

\appendixsection{Additional Tables}

Table~\ref{tab:appendix1} presents additional descriptive statistics not included in the main text.

\begin{table}[htbp]
\centering
\caption{Supplementary Descriptive Statistics}
\label{tab:appendix1}
\begin{tabular}{@{}lccc@{}}
\toprule
\tableheader{Variable} & \tableheader{Mean} & \tableheader{SD} & \tableheader{n} \\
\midrule
Variable A & 15.2 & 3.4 & 500 \\
\rowcolor{tablealt} Variable B & 22.8 & 5.1 & 498 \\
Variable C & 8.9 & 2.2 & 495 \\
\bottomrule
\end{tabular}
\end{table}

\appendixsection{Instruments}

Full text of instruments used in this study:

\begin{enumerate}
    \item \textbf{Instrument Name 1:} [Description or full text]
    \item \textbf{Instrument Name 2:} [Description or full text]
\end{enumerate}

\appendixsection{Additional Figures}

[Include any supplementary figures here]

% ============================================================================
% END DOCUMENT
% ============================================================================

\end{document}

